%%%%%%%%%%%%%%%%%%%%%%%%%%%%%%%%%%%%%%%%%
% Medium Length Professional CV
% LaTeX Template
% Version 2.0 (8/5/13)
%
% This template has been downloaded from:
% http://www.LaTeXTemplates.com
%
% Original author:
% Thanks : Rishi Shah 's Contribution
% inspired by his awesome contribution:
% https://www.overleaf.com/articles/rishi-shahs-resume/vgxvkmxktyxn
% Author : Allianzcortex
% contact me : github.com/Allianzcortex
% email : iamwanghz#gmail.com
%
% Important note:
% This template requires the resume.cls file to be in the same directory as the
% .tex file. The resume.cls file provides the resume style used for structuring the
% document.
%
%%%%%%%%%%%%%%%%%%%%%%%%%%%%%%%%%%%%%%%%%

%----------------------------------------------------------------------------------------
%	PACKAGES AND OTHER DOCUMENT CONFIGURATIONS
%----------------------------------------------------------------------------------------

\documentclass{resume} % Use the custom resume.cls style

\usepackage[left=0.40in,top=0.3in,right=0.75in,bottom=0.1in]{geometry} % Document margins
\usepackage{times}
\usepackage{hyperref}
\hypersetup{breaklinks=true}
\newcommand{\tab}[1]{\hspace{.2667\textwidth}\rlap{#1}}
\newcommand{\itab}[1]{\hspace{0em}\rlap{#1}}
\name{Joseph Voss} % Your name 
\address{ \href{https://josephvoss.com}{https://josephvoss.com} - jvoss@josephvoss.com - (512) 517-0648 }
%\address{123 Pleasant Lane \\ City, State 12345} % Your secondary addess (optional)
%\address{\faGithub{ github.com/harrypotter} \faLink{ harrypotter.me} \faEnvelope{ harrypotter@gmail.com}}

\begin{document}
%{\centerline {\em \textbf { Targeting a 4-month, full-time internship position from 2012.01-2012.05(delete it if needed) } } }
%----------------------------------------------------------------------------------------
%	EDUCATION SECTION
%----------------------------------------------------------------------------------------

\begin{rSection}{Education}

{\bf Bachelors of Science, Mechanical Engineering } \hfill { Aug 2014 - May 2018} 
\\{ { The University of Texas at Austin}} 

\end{rSection}

%-----------------------------------------------------------------------------------------------
%    Skills
%-----------------------------------------------------------------------------------------------
\begin{rSection}{Skills}

Python, Puppet, Golang, Kubernetes, Slurm, Bash, Git, eBPF, C++, Performance
  Tuning, GPFS, Lustre

% \begin{tabular}{ @{} >{\bfseries}l @{\hspace{6ex}} l }
% Programming: \ & aaa,bbb,ccc,ddd,eee,fff,ggg,hhh \\
% Software \& Tools: & {\textbf{Backend: }}qqqq,wwww,eeee,rrrr,tttt,yyyy,uuuu,iiii\\
% & {\textbf{FrontEnd: }}aaaaa,bbbbb,eeee,cccc\\
% & {\textbf{Others: }}uuuu,uyyy,oooo,aaaa,vvv,eeee
% \end{tabular}

\end{rSection}

%----------------------------------------------------------------------------------------
%	EXPERIENCE SECTION
%----------------------------------------------------------------------------------------
\begin{rSection}{Experience}

\href{http://olcf.ornl.gov}
{\textbf{HPC Systems Engineer, Oak Ridge National Laboratory}
  \hfill {\em Jun 2018 -- Present}}
  
\\- Provisioned Andes, new 700 node HPC cluster. Largest commodity cluster
    procured by ORNL to date.
\\- Wrote custom tool to boot HPC machines from container images. Transitioned
    several large scale systems to use it.
\\- Developed eBPF wrapper to compile and load Linux kernel profiling programs and
    output data to Kafka.
\\- Created Helm charts and automated pipelines to move system services to
    Kubernetes.
\\- Developed CI pipeline to stage Puppet changes on bare metal servers
\\- Reviewed proposals for new HPC Systems. Assisted in their provisioning and
    deployment.
\\- Extended Let's Encrypt Golang projects to create in-house certificate issuer for
    host authentication bootstrapping.
\\- Used Puppet to configure and manage large scale systems.
\\- Contributed to open source projects to improve system health and monitoring.
    Used to create load-balanced and always available data transfer cluster.
\\- Helped lead a team of student interns in the `19 Student Cluster Competition
\\- TLDR; Leveraged Puppet, Golang, Python, and Kubernetes to simplify
    management of HPC systems

\href{http://multimechanics.com}
{\textbf{DevOps Engineer, MultiMechanics}
  \hfill
  { \em Jan 2018 -- May 2018 }}
  
\\- Created automated build system using Vagrant. Converting software tools from
    Windows to Redhat and SUSE

\href{http://tacc.utexas.edu}
{\textbf{Student Intern, Texas Advanced Computing Center}}\hfill{\em Feb 2016-- Aug 2017}

\\- Developed an automated HPC testing harness using Jenkins, PyTest, and CMake that
    integrates with Slurm
\\- Created a heatmap visualization showing historical degradation and improvement in system performance
\\- Designed, built and managed a cluster of high performance compute nodes for the Student
    Cluster Competition
\\- Developed remote power monitoring system using SNMP, Graphite, and Grafana
\\- Attended Supercomputing Conference 2016 to compete with student teams from around the
    world, placed 4\textsuperscript{th} overall

\href{http://arlut.utexas.edu}
{\textbf{Science and Engineering Apprentice, Applied Research Laboratory}
  \hfill{\em May 2014 -- Aug 2015}}
\\- Created a suite of cross-compatible unit tests in C++ for open source software
\\- Redesigned the method of reading/writing out RINEX files to use OOP encapsulation
\\- Developed an inexpensive COTS GPS data collection platform using Python; decodes binary streams and writes out to a formatted RINEX file

\end{rSection}

%-----------------------------------------------------------------------------------------------
%    Publications
%-----------------------------------------------------------------------------------------------
\begin{rSection}{Publications}

\href{https://sc20.supercomputing.org/proceedings/sotp/sotp_pages/sotp107.html}
{\underline{Voss, J.} (2020). ``Anchor: Diskless Cluster Provisioning Using
Container Tools."
Presented at \textit{SC '20: International Conference for High Performance Computing,
Networking, Storage and Analysis}. Atlanta, GA, USA. \linebreak
https://sc20.supercomputing.org/proceedings/sotp/sotp_pages/sotp107.html}

\href{http://doi.acm.org/10.1145/3155105.3155106}
{\underline{Voss, J.}, Garcia, J. A., Proctor, W. C., \& Evans, R. T.
(2017). ``Automated System Health and Performance Benchmarking Platform."
In \textit{Supercomputing Conference '17: Proceedings of the 2nd international
HPC System Professionals Workshop at SC'17}. New York, NY, USA: ACM.
https://doi.acm.org/10.1145/3155105.3155106}

\href{https://doi.org/10.1016/j.parco.2017.07.002}
{Ababao, R., Garcia, J. A., \underline{Voss, J.}, Proctor, W. C., \& Evans, R.
T. (2017). ``Student Cluster \linebreak Competition 2016 reproducibility challenge: Genomic
partitioning with ParConnect." \textit{Parallel Computing.} \linebreak https://doi.org/10.1016/j.parco.2017.07.002}

\end{rSection}

%---
% Projects
%---
%\begin{rSection}{Open-source Projects}
%
%\href{https://github.com/olcf/greggd}
%  {Greggd}: Daemon to compile/load eBPF programs into the kernel and forward output
%to monitoring stack. https://github.com/olcf/greggd
%
%\href{https://github.com/olcf/anchor}
%  {Anchor}: Dracut module to enable diskless booting using squashfs and overlayfs
%https://github.com/olcf/anchor
%
%\end{rSection}

%----------------------------------------------------------------------------------------
%	Awards
%----------------------------------------------------------------------------------------

%\begin{rSection}{Achievements} \itemsep -3pt
%\item Presidential Achievement Scholar\hfill{ \em Aug 2014 }
%\item Terry Foundatation Scholar\hfill{ \em Aug 2014 }
%\item Member, Programmers in Science and Engineering\hfill{ \em Aug 2016 -- May 2018}
%\end{rSection}

%----------------------------------------------------------------------------------------
%	EXAMPLE SECTION
%----------------------------------------------------------------------------------------

% \begin{rSection}{Academic Achievements} 
%  Runners up in B.G.Shirke Vidyarthi Competition for Innovative Project organized by Pune Construction Engineering Research Foundation in January 2018
% \item Won First Prize in Model Making Competition Organized by Symbiosis Institute of Technology, Pune.
% \end{rSection}

% \newpage

% \end{rSection}

% \begin{rSection}{Personal Traits}
% \item Highly motivated and eager to learn new things.
% \item Strong motivational and leadership skills.
% \item Ability to work as an individual as well as in group.
% \end{rSection}
\end{document}
