% !TEX TS-program = xelatex
% !TEX encoding = UTF-8 Unicode
% -*- coding: UTF-8; -*-
% vim: set fenc=utf-8

%%%%%%%%%%%%%%%%%%%%%%%%%%%%%%%%%%%%%%%%%%%%%%%%%%%%%%%%%%%%%%%%%
%% SIMPLE-RESUME-CV
%% <https://github.com/zachscrivena/simple-resume-cv>
%% This is free and unencumbered software released into the
%% public domain; see <http://unlicense.org> for details.
%%%%%%%%%%%%%%%%%%%%%%%%%%%%%%%%%%%%%%%%%%%%%%%%%%%%%%%%%%%%%%%%%

% See "README.md" for instructions on compiling this document.

\documentclass[letterpaper,MMMyyyy,nonstopmode]{simpleresumecv}
% Class options:
% a4paper, letterpaper, nonstopmode, draftmode
% MMMyyyy, ddMMMyyyy, MMMMyyyy, ddMMMMyyyy, yyyyMMdd, yyyyMM, yyyy

%%%%%%%%%%%%%%%%%%%%%%%%%%%%%%%%%%%%%%%%%%%%%%%%%%%%%%%%%%%%%%%%%
%% PREAMBLE.
%%%%%%%%%%%%%%%%%%%%%%%%%%%%%%%%%%%%%%%%%%%%%%%%%%%%%%%%%%%%%%%%%

% CV Info (to be customized).
\newcommand{\CVAuthor}{Joseph Voss}
\newcommand{\CVTitle}{Joseph Voss' Resume}
\newcommand{\CVNote}{Compiled on {\today}}
\newcommand{\CVWebpage}{http://jvoss14.com}

% PDF settings and properties.
\hypersetup{
pdftitle={\CVTitle},
pdfauthor={\CVAuthor},
pdfsubject={\CVWebpage},
pdfcreator={XeLaTeX},
pdfproducer={},
pdfkeywords={},
unicode=true,
bookmarks=true,
bookmarksopen=true,
pdfstartview=FitH,
pdfpagelayout=OneColumn,
pdfpagemode=UseOutlines,
hidelinks,
breaklinks}

% Shorthand.
\newcommand{\Code}[1]{\mbox{\textbf{#1}}}
\newcommand{\CodeCommand}[1]{\mbox{\textbf{\textbackslash{#1}}}}

%%%%%%%%%%%%%%%%%%%%%%%%%%%%%%%%%%%%%%%%%%%%%%%%%%%%%%%%%%%%%%%%%
%% ACTUAL DOCUMENT.
%%%%%%%%%%%%%%%%%%%%%%%%%%%%%%%%%%%%%%%%%%%%%%%%%%%%%%%%%%%%%%%%%

\begin{document}

%%%%%%%%%%%%%%%
% TITLE BLOCK %
%%%%%%%%%%%%%%%

\Title{\CVAuthor}

\begin{SubTitle}
{5610 Abilene Trail, Austin, TX 78749, USA}
\par
\href{\CVWebpage}
{\url{\CVWebpage}}
\,\SubBulletSymbol\,
+1\,(512)\,517-0468
\,\SubBulletSymbol\,
\href{mailto:josephvoss14@gmail.com}
{josephvoss14@gmail.com}

\end{SubTitle}

\begin{Body}

%%%%%%%%%%%%%%%
%% EDUCATION %%
%%%%%%%%%%%%%%%

\Section
{Education}
{Education}
{PDF:Education}

\Entry
%\href{http://www.example.com/my-university}
{\textbf{Bachelor of Science, Mechanical Engineering}},
University of Texas at Austin
\hfill
\DatestampYMD{2014}{08}{25} --
\DatestampYMD{2018}{05}{04}

\Gap
\begin{center}
Related Courses
\end{center}
\Gap
\begin{Detail}
Advanced Mechatronics II, Parallel Computing, Programming and Engineering Computational Methods, Heat Transfer, Engineering Vibrations, Machine Elements, Material Engineering, Fluid Mechanics, Thermodynamics, Solids, Statics, Engineering Design and Graphics, Differential Equations, Matrices and Matrix Calculations, Engineer Statistics, Engineering Communication
\end{Detail}

\BigGap
\Entry
\href{http://www.example.com/my-college}
{\textbf{Study Abroad}},
IES Abroad: Vienna, Austria
\hfill
\DatestampYMD{2015}{05}{22} -- 
\DatestampYMD{2015}{06}{22}

%%%%%%%%%%%%%%%%%%%%%%%%%
%% RESEARCH EXPERIENCE %%
%%%%%%%%%%%%%%%%%%%%%%%%%

\Section
{Experience}
{Experience}
{PDF:Experience}

\Entry
\href{http://tacc.utexas.edu}
{\textbf{Texas Advanced Computing Center}}

\Gap
\BulletItem
Student Intern, High Performance Computing
\hfill
\DatestampYMD{2017}{07}{01} --
\DatestampYMD{2017}{08}{25}
\begin{Detail}
\SubBulletItem
Developed an automated HPC testing harness using Jenkins, PyTest, and CMake \newline
that integrates seamlessly with SLURM
\SubBulletItem
Created a heatmap visualization using Bokeh, showing 
degredation  and improvement \newline in system performance
\SubBulletItem
Submitted a research paper describing the test harness developed\newline
to the HPC System Professionals Workshop at Supercomputing Conference 17.
\end{Detail}

\Gap
\BulletItem
Team Member, Student Cluster Competition
\hfill
\DatestampYMD{2016}{02}{20} --
\DatestampYMD{2017}{03}{30}
\begin{Detail}
\SubBulletItem
Designed, built and managed a cluster of high performance compute nodes
\SubBulletItem
Developed remote power monitoring system using SNMP, Graphite, and Grafana
\SubBulletItem
Learned how to use and profile several HPC applications%: namely Linpack, HPCG, Paraview, and Hashcat
\SubBulletItem
Attended Supercomputing Conference 2016 to compete with student teams from around \newline 
the world, placed 4\textsuperscript{th} overall
\SubBulletItem
Published a reproducibility study to the Parallel Computing journal.
\end{Detail}

\BigGap

\Entry
\href{http://tridentresearch.com/}
{\textbf{Trident Research LLC}}

\Gap
\BulletItem
Mechanical Engineer Intern
\hfill
\DatestampYMD{2016}{06}{25} --
\DatestampYMD{2016}{08}{14}
\begin{Detail}
\SubBulletItem
Designed and assembled charging system for naval buoys
\SubBulletItem
Created drawings and 3D models in Solidworks of custom parts
\SubBulletItem
Wrote embedded firmware for safe charging of buoys
\SubBulletItem
Completed acceptance testing for both custom and COTS parts
\SubBulletItem
Wrote and updated documentation of the naval buoy system
\end{Detail}

\BigGap

\Entry
\href{http://arlut.utexas.edu}
{\textbf{Applied Research Laboratory}}

\Gap
\BulletItem
Student Technician, Space and Geophysics Lab
\hfill
\DatestampYMD{2015}{07}{15} --
\DatestampYMD{2015}{08}{20}
\begin{Detail}
\SubBulletItem
Redesigned the method of reading/writing out RINEX files to use the OOP principle of \newline encapsulation
\SubBulletItem
Updated the in-house code base to use the new RINEX objects for file I/O
\SubBulletItem
Extensive cataloging of the applications within the in-house code-base
\end{Detail}

\Gap
\BulletItem
Student Technician, Space and Geophysics Lab
\hfill
\DatestampYMD{2015}{01}{16} --
\DatestampYMD{2015}{05}{05}
\begin{Detail}
\SubBulletItem
Created a suite of cross-compatible unit tests in C++
\SubBulletItem
Helped develop in-house testing framework
\SubBulletItem
Wrote documentation for how later unit testing should be executed
\end{Detail}

\Gap
\BulletItem
Science and Engineering Apprentice, Space and Geophysics Lab
\hfill
\DatestampYMD{2014}{05}{20} --
\DatestampYMD{2014}{08}{04}
\begin{Detail}
\SubBulletItem
Developed an inexpensive COTS GPS data collection platform using Python
\SubBulletItem
Wrote software capable of decoding binary streams, translating them to the floating point \newline
representation, and writing out to formatted RINEX file
\SubBulletItem
Interfaced with GPS receiver mounted on a DIP via serial communication
\end{Detail}


%%%%%%%%%%%%
%% SKILLS %%
%%%%%%%%%%%%

\Section
{Skills}
{Skills}
{PDF:Skills}

\Entry
Solidworks,
C++, 
Python, 
Git, 
Bash,
CMake,
Jenkins,
Linux management \& development, 
Soldering,
MATLAB,
\LaTeX,
Microsoft Word,
Microsoft Excel,
Basic machining and assembly experience.

%%%%%%%%%%%%%%%%%%%%%%%%%%%%%%%%%%%%%%%%%%%%
%% Publications %%
%%%%%%%%%%%%%%%%%%%%%%%%%%%%%%%%%%%%%%%%%%%%

\Section
{Publications}
{Publications}
{PDF:Publications}

\Entry
\underline{Voss, J.}, Garcia, J. A., Proctor, W. C., \& Evans, R. T.
(Submitted). Automated System Health and Performance Benchmarking Platform.
In \textit{Supercomputing Conference ’17: Proceedings of the 2nd international
HPC System Professionals Workshop at SC’17}. New York, NY, USA: ACM.
\Gap

\Entry
\href{https://doi.org/10.1016/j.parco.2017.07.002}
Ababao, R., Garcia, J. A., \underline{Voss, J.}, Proctor, W. C., \& Evans, R. T. (2017). "Student Cluster Competition 2016 reproducibility challenge: Genomic partitioning with ParConnect." \textit{Parallel Computing.} https://doi.org/10.1016/j.parco.2017.07.002

%%%%%%%%%%%%%%%%%%%%%%%%%%%%%%%%%%%%%%%%%%%%
%% PROFESSIONAL ACHIEVEMENTS %%
%%%%%%%%%%%%%%%%%%%%%%%%%%%%%%%%%%%%%%%%%%%%

\Section
{Professional Achievements}
{Professional Achievements}
{PDF:ProfessionalAchievements}

\Entry
Terry Foundation Scholar
\hfill
\DatestampY{2014}--
Current

\Entry
Eagle Scout, Troop 3
\hfill
\DatestampY{2012}

\Entry
Presidential Achievement Scholar
\hfill
\DatestampY{2014}--
Current

%\Entry
%Member, Programmers in Science and Engineering
%\hfill
%\DatestampY{2016}--
%Current
\end{Body}

%%%%%%%%%%%
% CV NOTE %
%%%%%%%%%%%

%\UseNoteFont%
%\null\hfill%
%[\textit{\CVNote}]

\end{document}
